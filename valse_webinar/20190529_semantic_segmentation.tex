\ifx\allfiles\undefined
\documentclass[UTF8]{ctexart}

\usepackage{graphicx}
\usepackage{pythonhighlight}

\bibliographystyle{plain}%指定参考文献的排版样式

\begin{document}
\title{20190529-12 图像语义分割-理解每个像素}
\author{主持人:程明明(南开大学 )}
\date\today
\maketitle
\tableofcontents
%\else
%\section{20190529-12 图像语义分割-理解每个像素}
\fi
\section{Context For Semantic Segmentation}

\subsection{介绍}
\paragraph{报告嘉宾}
俞刚(旷视科技)

\paragraph{报告时间}
2019年5月29日(星期三)晚上20:30(北京时间)

\paragraph{报告题目}Context For Semantic Segmentation

\paragraph{报告人简介}
俞刚博士现为旷视科技研发总监,Detection 组负责人,2014年毕业于新加坡南洋理工大学。
博士毕业后在南洋理工大学从事 research fellow 的研发工作。2014 年底加入旷视科技公司。
其研究方向主要集中在计算机视觉以及机器学习方面,包括物体检测,语义分割,行人姿态估计以及人体动作行为分析。
已经在顶级会议以及期刊上面发表学术论文二十余篇。同时著有书籍一本。
俞刚博士带队参加 2017 COCO+Places 挑战赛获得检测第一名,人体姿态估计第一名;接着带队参加 2018 COCO+Mapillary 挑战赛,获四项第一。
个人主页\cite{Gangyu}。

\paragraph{报告摘要}
Semantic segmentation is a fundamental problem in computer vision society. 
The main challenge is how to deal with the ambiguities when the pixels have similar appearance but different categories. 
Context information is an important cue to deal with the problem. 
In this talk, I will provide four approaches to model the context information in the image, covering the backbone, head and loss in the network design. 
Competitive results have been reported based on the semantic segmentation benchmarks.

\section{Pixel-Level Image Understanding with Semantic Segmentation and Panoptic Segmentation}

\subsection{介绍}

\paragraph{报告嘉宾}赵恒爽 (The Chinese University of Hong Kong)

\paragraph{报告时间}2019年5月29日(星期三)晚上20:00(北京时间)

\paragraph{报告题目}Pixel-Level Image Understanding with Semantic Segmentation and Panoptic Segmentation

\paragraph{报告人简介}
Hengshuang Zhao is currently a fourth year Ph.D. student at The Chinese University of Hong Kong, supervised by Prof. Jiaya Jia. 
Before that, he received the B.E. degree from Huazhong University of Science and Technology in 2015. 
His general research interests cover the broad area of computer vision and deep learning, with special emphasis on high-level image recognition and pixel-level image understanding. 
He and his team won 1st places in ImageNet Scene Parsing Challenge 2016, LSUN Semantic Segmentation Challenge 2017 and WAD Drivable Area Segmentation Challenge 2018. 
Part of his research projects are supported by SenseTime, Adobe, Uber and Intel. His works have been cited for about 1200 times.
个人主页\cite{Hengshuang}。

\paragraph{报告摘要}
Pixel-Level image understanding is a fundamental while challenging task in computer vision. It predicts dense values for all pixels in the image, and is regarded as a very important task that can help achieve a deep understanding of scene, objects, and human. In this talk, I will mainly cover the topics of semantic segmentation and panoptic segmentation. For each topic, I will first go through recent deep learning based approaches and then detail our latest efforts with Point-wise Spatial Attention Network (PSANet) for semantic segmentation and Unified Panoptic Segmentation Network (UPSNet) for panoptic segmentation. Finally, I will discuss some existing problems and the remaining challenges.

\renewcommand\refname{参考文献}
\begin{thebibliography}{99}
    \bibitem[1]{GCN} Large Kernel Matters -- Improve Semantic Segmentation by Global Convolutional Network, Chao Peng, Xiangyu Zhang, Gang Yu, Guiming Luo, Jian Sun,CVPR, 2017
    \bibitem[2]{DFN} Learning a Discriminative Feature Network for Semantic Segmentation, Changqian Yu, Jingbo Wang, Chao Peng, Changxin Gao, Gang Yu, Nong Sang, CVPR, 2018
    \bibitem[3]{BiSeNet} BiSeNet: Bilateral Segmentation Network for Real-time Semantic Segmentation, Changqian Yu, Jingbo Wang, Chao Peng, Changxin Gao, Gang Yu, Nong Sang, ECCV, 2018
    \bibitem[4]{Panoptic_Megvii} http://presentations.cocodataset.org/ECCV18/COCO18-Panoptic-Megvii.pdf
    \bibitem[5]{pspnet} Pyramid Scene Parsing Network, CVPR 2017. Hengshuang Zhao, Jianping Shi, Xiaojuan Qi, Xiaogang Wang, Jiaya Jia.
    \bibitem[6]{ICNet} ICNet for Real - Time Semantic Segmentation on High - Resolution Images, ECCV 2018. Hengshuang Zhao, Xiaojuan Qi, Xiaoyong Shen, Jianping Shi, Jiaya Jia.
    \bibitem[7]{PSANet}PSANet: Point - wise Spatial Attention Network for Scene Parsing, ECCV 2018. Hengshuang Zhao, Yi Zhang, Shu Liu, Jianping Shi, Chen Change Loy, Dahua Lin, Jiaya Jia.
    \bibitem[8]{UPSNet} UPSNet: A Unified Panoptic Segmentation Network, CVPR 2019. Yuwen Xiong, Renjie Liao, Hengshuang Zhao, Rui Hu, Min Bai, Ersin Yumer, Raquel Urtasun.
    \bibitem[9]{Hengshuang} https://hszhao.github.io/
    \bibitem[10]{Gangyu} http://www.skicyyu.org/
    \bibitem[11]{slides} http://valser.org/article-320-1.html
\end{thebibliography}
\bibliography{BiBTex}

\ifx\allfiles\undefined
\end{document}
\fi